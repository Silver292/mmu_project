\begin{minted}[linenos,tabsize=2,breaklines]{python}
#!/usr/bin/env python3
"""
Script to get sensor data from DHT22 and print it to console
"""

__author__ = "Tom Scott"

import Adafruit_DHT

_SENSOR = Adafruit_DHT.AM2302
_GPIO_PIN = 4

class SensorData(object):
    """
    Data object containing humidity and temperature data.
    """

    def __init__(self, humidity, temperature):
        """
        Set temperature and humidity
        """
        self.humidity = humidity
        self.temperature = temperature

    def has_data(self):
        """
        Returns True if humidity and temperature are not None.
        """
        return self.humidity is not None and self.temperature is not None

def get_sensor_data():
    """
    Returns a SensorData object containing the latest data from the DHT22 sensor.
    The method will try 15 times to get data, if there is still no data available
    it will return an empty SensorData object
    """

    # Try to grab a sensor reading.  Use the read_retry method which will retry up
    # to 15 times to get a sensor reading (waiting 2 seconds between each retry).
    humidity, temperature = Adafruit_DHT.read_retry(_SENSOR, _GPIO_PIN)

    return SensorData(humidity, temperature)

def main():
    """ Main entry point of the app """

    # Note that sometimes you won't get a reading and
    # the results will be null (because Linux can't
    # guarantee the timing of calls to read the sensor).
    sensor_data = get_sensor_data()

    if sensor_data.has_data():
        # Format and display data to console
        print('Temp={0:0.1f}*  Humidity={1:0.1f}%'.format(sensor_data.temperature, sensor_data.humidity))


if __name__ == "__main__":
    """ This is executed when run from the command line """
    main()
\end{minted}