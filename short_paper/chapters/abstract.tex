This paper describes the development of an \gls{iot} monitoring system using
ThingsBoard \gls{iot} platform.
ThingsBoard is an open source software tool, which is used to collect,
monitor and visualise streams of data received in real-time by sensor devices.
The platform can be hosted in the cloud and provides 
\gls{mqtt}, \gls{coap} and \gls{http} protocols support.
\gls{mqtt} and \gls{http} protocols have mostly been used to develop various \gls{iot} systems.
However, this paper investigates the use of the \gls{coap} in transmitting sensor data to the cloud.
It aims to explore how \gls{coap} fits into the \gls{iot} ecosystem and what advantages it offers over other \gls{iot} protocols.
A \gls{coap}--based \gls{iot} architecture is proposed using a \gls{rpi} and sensors acting as \gls{iot} endpoints.
These endpoints will poll sensors (e.g. temperature and humidity) and using \gls{coap} 
will send the latest data formatted as \gls{json} to the ThingsBoard cloud endpoint at regular intervals.
ThingsBoard can create real-time \gls{iot} Dashboards for sensors data visualization and share it with users.