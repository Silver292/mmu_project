This research investigates the use of \gls{coap} in transmitting sensor data
to the cloud. It aims to explore how \gls{coap} fits into the \gls{iot} ecosystem and
what advantages it offers over other \gls{iot} protocols. It also shows how a \gls{rpi} can
be used with Python to create an \gls{iot} device capable of transferring data to the cloud. 

The \gls{rpi} has been connected to a DHT22 temperature and humidity sensor using
the \glspl{rpi} \gls{gpio} pins. A Python script running on the \gls{rpi} uses
an external library provided by AdaFruit to poll the sensor at intervals and 
obtain the current temperature and humidity. Using the CoAPthon \citep{tanganelli_coapthon:_2015} 
library, a \gls{coap} client is created and used to send a \gls{coap} POST 
message containing the sensor data to a ThingsBoard 
\citep{thingsboard_inc._thingsboard_2018} telemetry endpoint. This data 
is then displayed to the user in a dashboard.
