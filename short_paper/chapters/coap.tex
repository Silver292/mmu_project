\acrfull{coap} is a transfer protocol specialised for use with the web, constrained
nodes and constrained networks \citep{shelby_constrained_2014}. The protocol is 
designed for \gls{m2m} applications and is ideally suited for use within the 
\gls{iot} ecosystem. \glspl{coap} features of observable resources, multicasting, \gls{m2m} discovery 
make it a better fit for \gls{iot} applications than \gls{http} \citep{kovatsch_californium_2014}.

\gls{coap} recognises that web services have become dependent on \gls{rest} 
architecture and works to implement a subset of \gls{rest} common with HTTP while
optimising for \gls{m2m} applications \citep{shelby_constrained_2014}. It achieves 
this by offering built-in discovery, multicast support and asynchronous message 
exchanges \citep{shelby_constrained_2014}. 

\gls{coap} uses a compact binary format with a fixed header size of 4 bytes, 
exchanging messages over \gls{udp} or \gls{dtls} to send messages securely. \gls{coap} 
resources are addressable by \glspl{uri} and can be interacted with through the 
same methods as \gls{http}: GET, PUT, POST and DELETE.

With regard to reliability, \gls{coap} offers four types of messages: Confirmable, 
Non-Confirmable, Acknowledgement and Reset \citep{bellavista_towards_2016}.
After a Confirmable request is sent to a \gls{coap} endpoint, the endpoint will 
respond with an Acknowledgement message. This message can contain the requested 
data in a `piggybacked' response. Otherwise, an empty Acknowledgement message is sent 
and a Confirmable message will be sent once the data is ready. The original
requester will then respond with an empty Acknowledgement message to confirm receipt 
of the data \citep{shelby_constrained_2014}.
