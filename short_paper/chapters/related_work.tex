\citet{rode_iot_2017} carries out a similar investigation, using a \gls{rpi} 
device as an \gls{iot} node connected to sensors. The \gls{rpi} collects the 
data from the sensors and then transmits the data to a cloud platform. The 
cloud platform in this instance is a \gls{http} server which will receive the
sensor data and display it to the user. \citet{rode_iot_2017} proposes using the 
\gls{mqtt} protocol to transmit the sensor data from the \gls{rpi} to the
\gls{http} server.
\gls{mqtt} is a popular \gls{iot} protocol developed to specialise in the transfer
of data from \glspl{wsn} \citep{hunkeler_mqtt-s_2008}. \gls{mqtt} works on a 
publish/subscribe model; in \citet{rode_iot_2017} the \gls{rpi} acts a \textit{publisher}, 
publishing the sensor data to the broker and the \gls{http} server acts as a
\textit{subscriber}. The \gls{mqtt} broker is responsible for coordinating subscribers 
to the data and subscribers will usually have to contact the broker explicitly in 
order to subscribe \citep{hunkeler_mqtt-s_2008}.
This contrasts to the approach taken in this proposal using \gls{coap}, where each
node in the \gls{coap} network acts as both a server and a client in a more traditional
\gls{http} model and nodes within the infrastructure will communicate with one
another directly.

\citet{jassas_smart_2015} used a \gls{rpi} connected to sensors to measure 
patients' body temperature and transmit this data wirelessly to the cloud.
In that paper, the data was transmitted to an \gls{aws} cloud computing platform.
There the data was stored, mined in order to make decisions, 
and displayed to the user allowing the data to be updated and reviewed.
The data was transmitted from the \gls{rpi} to the \gls{aws} server using \gls{ssl}.
The development of specialised protocols for constrained devices, such as \gls{coap}
could allow these health monitoring \glspl{rpi} to save power,
save network bandwidth and potentially receive more readings to process.

\citet{lee_internet_2018} used a \gls{rpi} combined with a DHT22 sensor to measure
the indoor temperature in real-time. This data was transmitted using \gls{http}
to a \gls{rest} \gls{api}, where the temperature was stored in a database.
These temperatures were then used to inform an application replicating 
the actions of an air conditioner. The use of a \gls{rest}ful \gls{api} in 
this paper would allow the project to easily be adapted to using \gls{coap} 
to replace \gls{http}.