\documentclass[11pt]{article}
\usepackage{float}
\usepackage{graphicx}
\usepackage{lscape}
\usepackage{pdflscape}
\usepackage{booktabs}
\usepackage{natbib}
\usepackage{pdfpages}
\usepackage[acronym, toc, nonumberlist]{glossaries}
\usepackage{url} 
\usepackage{hyperref}
\usepackage{xcolor}
\hypersetup{
    colorlinks,
    linkcolor={red!50!black},
    citecolor={blue!50!black},
    urlcolor={blue!80!black}
}
% \usepackage[none]{hyphenat}

\newcommand{\figref}[1]{\hyperref[#1]{\figurename~\ref*{#1}}}


\makeglossaries
\newacronym{iot}{IoT}{Internet of Things}
\newacronym{coap}{CoAP}{The Constrained Application Protocol}
\newacronym{udp}{UDP}{User Datagram Protocol}
\newacronym{rest}{REST}{Representational State Transfer}
\newacronym{m2m}{M2M}{Machine-to-Machine}
\newacronym{rpi}{RPi}{Raspberry Pi}
\newacronym{json}{JSON}{JavaScript Object Notation}
\newacronym{gpio}{GPIO}{General Purpose Input Output}

\title{CoAP based IoT data transfer from a Raspberry Pi to Cloud}
\author{Thomas Scott}
\date{20th November 2018}

\begin{document}
	\pagenumbering{gobble}
	\maketitle
	\newpage
	\pagenumbering{arabic}

	\section{Abstract}
	This research investigates the use of \gls{coap} in transmitting sensor data
to the cloud. It aims to explore how \gls{coap} fits into the \gls{iot} ecosystem and
what advantages, if any, it offers over other \gls{iot} protocols. A framework is proposed 
using a \gls{rpi} and sensor acting as an \gls{iot} endpoint. This endpoint will allow for
\gls{coap} requests and will poll the sensor and return the latest data as \gls{json}.
The endpoint will be polled from a cloud service, this service will then display the data to 
the user.

	\section{Introduction}
	The reduced cost of low powered small devices, such as the \gls{rpi}, has made it more accessible to create bespoke systems. 
This combined with the increasing popularity of home automation allows for these devices to be used in the \gls{iot}.

The \gls{iot} can be viewed as a large distributed network comprising of highly dynamic devices \citep{miorandi2012internet}. Small low powered ``smart'' devices can connect and communicate with one another. Some of these devices can contain or communicate with sensors that record real world data. This data can then be transmitted to other devices allowing them to trigger actions. In this way groups of smart devices can be used to improve day to day situations such as automated houses (thermostats and heating etc.), security and improved monitoring.

The Raspberry Pi \citep{pi3model} is a credit card sized computer developed by the Raspberry Pi Foundation. 
The \gls{rpi}'s ability to act as a GNU/Linux server and the interfacing services provided by its general purpose I/O pins make it a popular 
choice of hardware for \gls{iot} applications. \citep{kumar2016iot}

With 48\% of the UK market considering their smartphone as the most important device for internet access \citep{ofcom2018} allowing users to use their handheld devices to view and manage their data has become increasingly necessary. Cloud platforms that allow access from any device go a long way to 
solving this problem. Storing sensor data in the cloud allows for easy access to users from any device as well as allowing for scalable storage.

As these devices are limited in computing power it is important that the devices communicate efficiently. 
This paper explores the use of \gls{coap} as a protocol to transmit sensor data from a small, low powered device (\gls{rpi}) to send sensor data to the cloud.

In this system the a sensor will be attached to the \gls{rpi}, the \gls{rpi} will be responsible for taking the data from the sensor and then 
using the \gls{coap} protocol to transmit this data to the cloud platform.

	\section{Design}
	The system shall consist of four main elements: the sensor, the \gls{rpi}, \gls{coap} and the cloud platform.
The sensor will collect the data and pass this to the \gls{rpi}. The \gls{rpi} will then be responsible for manipulating
the data into a suitable format for transmission via \gls{coap}. The implementation of \gls{coap} will communicate with
the cloud platform. The cloud platform will store the data, allowing access to users.

\begin{figure}[H]
    \centering
    \makebox[1\textwidth]{\includegraphics[width=1\textwidth]{assets/Project_Framework.png}}
    \caption{\label{fig:proj_framework} The project infrastructure.}
\end{figure}


The DHT22 sensor will connect to the \gls{rpi} using the \gls{rpi}'s on board \gls{gpio} ports as shown in \figref{fig:rpi_wiring}. 
Using the AdaFruit Python DHT library, a library that provides methods to interact with DHT sensors connected to the \gls{rpi}'s \gls{gpio} pins, 
and the CoAPthon library, a Python implementation of the \gls{coap} protocol, a Python script will create a \gls{coap} endpoint. This endpoint will expose 
the DHT methods for getting the latest data. When this endpoint receives a request the data will be current temperature and humidity will be
retrieved from the sensor. Once this data has been collected it will be formatted in to \gls{json} using Python. This formatted data will then
be the payload for the response. The cloud service will be responsible for displaying the data from the sensor and making the request to the 
\gls{rpi} to get up to date information. This process is shown in \figref{fig:data_flow}.


\begin{figure}[H]
    \centering
    \makebox[1\textwidth]{\includegraphics[width=1\textwidth]{assets/rpi_wiring.png}}
    \caption{\label{fig:rpi_wiring} Wiring diagram for connecting the sensor to the \gls{rpi}.}
\end{figure}

\begin{figure}[H]
    \centering
    \makebox[1\textwidth]{\includegraphics[width=0.5\textwidth]{assets/data_flow.png}}
    \caption{\label{fig:data_flow} Shows the flow of data from sensor to user.}
\end{figure}


	\section{Methodology}
	The aim of the proposed system is to investigate the implementation of \gls{coap} 
on a \gls{rpi} and how \gls{coap} can be used to transmit data to the cloud. 

To achieve this a \gls{coap} endpoint will need to be created on the \gls{rpi}. 
The \gls{rpi} will collect sensor data at intervals and store them 
locally on the device. The \gls{rpi} will act as a \gls{coap} server that will 
send to \gls{rest} POST requests with the sensor data and the time
the reading was taken in a \gls{json} format.

The clouds responsibility will be to receive the POST requests from the \gls{coap} 
endpoint hosted on the \gls{rpi} and to store and format the data. 
The cloud should send an Acknowledgement message to the \gls{coap} endpoint, 
containing a 2.01 (Created) Response Code or a 2.04 (Changed) Response Code 
and the URI of the created / updated resource \citep{shelby_constrained_2014}. 

With \gls{coap} endpoints acting as both a client, that sends requests, and a 
server implementation of \gls{coap} will be needed 
in each the \gls{rpi} and the cloud platform. The \gls{rpi} will then regularly 
retrieve readings from the sensor and send a POST request 
at intervals to the cloud \gls{coap} endpoint. 

\begin{figure}[H]
    \centering
    \makebox[1\textwidth]{\includegraphics[width=1\textwidth]{assets/rpi_cloud_communication.png}}
    \caption{\label{fig:rpi_cloud_comms} Diagram showing the communication between \gls{rpi} and the Cloud.}
\end{figure}

	\section{Software and Hardware specification}
	\subsection{Hardware}
\begin{enumerate}
    \item Raspberry Pi 3

        Raspberry Pi 3 Model B, includes built in WiFi, \gls{gpio} ports and a 1.2GHz Quad-Core processor.
    \item Micro SD card
    
        Used to load operating system onto the \gls{rpi}.
    \item Power cord
    
        Supplies power to the \gls{rpi}.
    \item DHT22 temperature and humidity sensor
    
        Connects to the \gls{rpi} using the \gls{rpi}'s \gls{gpio} ports. Will be used to provide data.

\end{enumerate}

\subsection{Software}
\begin{enumerate}
    \item Python 3

        The Python programming language will be used to create the scripts and software needed on the \gls{rpi}.
        This is due to the languages popularity when creating projects on the \gls{rpi} and the languages wide selection
        of networking packages.
    \item CoAPthon
    
        Python implementation of \gls{coap}. Licensed under the MIT license. \href{https://github.com/Tanganelli/CoAPthon}{Github}.

    \item AdaFruit Python DHT
        
        Python library to retrieve sensor data from the DHT22. \href{https://github.com/adafruit/Adafruit_Python_DHT}{Github}.

    \item Git Version Control
    
        Source code version control system to allow for adjustments to the code.

    \item Cloud Platform
    
        Will act as an end point to the \gls{rpi} where the sensor data will be stored.
\end{enumerate}

	\section{Conclusion}
	This system looks into the use of \gls{coap} with regards to transmitting sensor data using a \gls{rpi}.
The data is stored on a cloud platform to provide ease of accessability to the user.

	\printglossaries

	\bibliographystyle{agsm}
	\bibliography{references}
\end{document}

